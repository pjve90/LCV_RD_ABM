\documentclass{article}
\usepackage[utf8]{inputenc}
\usepackage[margin=2.6cm]{geometry}
\usepackage{caption}
\usepackage[round]{natbib}
\usepackage{setspace}
\onehalfspacing

\title{The role of resource dynamics in the variability of life cycles within a female human population.}
\author{Pablo J. Varas Enriquez, Monique Borgerhoff Mulder, Heidi Colleran,
\\
Daniel Redhead, Dieter Lukas}
\date{\today}

\begin{document}

\maketitle

\begin{abstract}
    The female human life cycle is characterised by a long lifespan with a typically short reproductive career, which is between long juvenile and post-reproductive stages. There is a high variability in longevity, and the timing and outcome of reproduction. This variability may reflect a life history trade-off between survival and fertility. Formal theoretical models have proposed that the surplus resources produced during adulthood, and inter-generational resource transfers towards juveniles, have driven the evolution of the female human life cycle. However, it remains unclear as to how and whether variation in production and resource sharing (i.e. resource dynamics) shape the distribution of core features of the female human life cycle within a population. Here, we develop a theoretical framework to describe how different resource dynamics influence the variability of the female human life cycle. For this, we build a computational model to assess how the structure of resource dynamics at different stages of the life cycle influences the variability of life history traits of a population. For the structure of resource dynamics, we include a sub-model within our simulation that dictate stochasticity in resource production, and further  model the network structure of resource sharing by implementing a stage-structured stochastic blockmodel. Regarding the life history traits, the allocation of resources towards them in our simulation is deterministic and is based on surpassing the amount of resources set as thresholds for survival and reproductive costs. We expect that differences in resource production and sharing, and the allocation of such resources towards survival and reproduction, across the lifespan is important for understanding the mechanisms that drive the demographic diversity that is observed among human populations.
    \end{abstract}

\end{document}