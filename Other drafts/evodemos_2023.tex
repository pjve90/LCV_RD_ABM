\documentclass{article}
\usepackage[utf8]{inputenc}
\usepackage[margin=2.6cm]{geometry}
\usepackage{caption}
\usepackage[round]{natbib}
\usepackage{setspace}
\onehalfspacing

\title{The role of individual resource dynamics in the variability of life cycles within a female human population.}
\author{Pablo J. Varas Enriquez, Monique Borgerhoff Mulder, Heidi Colleran,
\\
Daniel Redhead, Dieter Lukas}
\date{\today}

\begin{document}

\maketitle

\begin{abstract}
    The female human life cycle is characterized by a long lifespan, within which there is usually a short reproductive period in between long juvenile and post-reproductive stages. Within those boundaries, there is a high variability in the length of longevity, and the timing and outcome of reproduction, usually associated with a life history trade-off between survival and fertility. The evolution of the female human life cycle has been related to the surplus of resource production during adulthood and the inter-generational transfers towards juveniles. However, there is no clear understanding whether and how these resource dynamics shape the observed variability within human populations. Here we develop a theoretical framework to describe how different resource dynamics influence the variability of the female human life cycle. For this, an agent-based model is used to structure and describe how resource dynamics (i.e. production, sharing, and storage) at different developmental stages of individuals can lead to different variability levels within a population. Furthermore, the model is set under multiple combinations of production and sharing probabilities (i.e. high, medium, and low) in order to assess how resource dynamics play different roles in the variability of life cycles among individuals. We expect that differences in the ways in which individuals obtain and allocate resources across their lifespan can contribute to understand the mechanisms behind the demographic diversity observed among human populations, as well as offer a tool to further analyze the link between resource availability and the female human life cycle. 
\end{abstract}

\end{document}