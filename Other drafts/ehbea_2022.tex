\documentclass{article}
\usepackage[utf8]{inputenc}
\usepackage[margin=2.6cm]{geometry}
\usepackage{caption}
\usepackage[round]{natbib}
\usepackage{setspace}
\onehalfspacing

\title{The role of resource dynamics in variations of the life cycle among female humans}
\author{Pablo J. Varas Enriquez, Monique Borgerhoff Mulder, Heidi Colleran, Dieter Lukas}
\date{\today}

\begin{document}

\maketitle

\begin{abstract}
    The female human life cycle is characterized by a long lifespan, within which there is usually a short reproductive period in between long juvenile and post-reproductive stages. Within those boundaries, there is a high variability in the timing and outcome of longevity and reproduction, with individuals showing short lifespans with high fertility and vice versa. The evolution of the female human life cycle has been related to the surplus of resource production during adulthood and the inter-generational transfers towards juveniles. However, there is no clear understanding whether and how these resource dynamics shape the observed variation in human populations. Here we develop a theoretical framework to describe how different resource dynamics influence the variability of the female human life cycle. For this, an agent-based model is used to structure and describe how resource dynamics (i.e. production, consumption, sharing, and storage) at different developmental stages of individuals can lead to different variability levels within a population. We expect that differences in the ways in which individuals obtain and allocate resources across their lifespan can contribute to understand the boundaries under which the female human life cycle varies, allowing to understand other possible scenarios beyond the common longevity versus fertility trade-off.
\end{abstract}

\end{document}