\documentclass{article}
\usepackage[utf8]{inputenc}
\usepackage[margin=2.6cm]{geometry}
\usepackage{float}
\usepackage{rotating}
\usepackage{caption}
\usepackage[round]{natbib}
\usepackage{setspace}
\onehalfspacing

\title{The interplay of resources and sociality explain the diversity of female human life-history strategies.}
\author{Pablo J. Varas Enríquez, Dieter Lukas, Heidi Colleran, Monique Borgerhoff Mulder $^*$\\\\
$*$ No special order of authors}
\date{\today}

\begin{document}

\maketitle

\tableofcontents

\begin{abstract}
    The female human life cycle has evolved to have a long lifespan with a short reproductive period in between long juvenile and post-reproductive stages. Within these boundaries, humans can present diverse life-history strategies that can go from long lifespans and low reproductive output (e.g. post-demographic transition Japan) to short lifespans and high number of offspring (e.g. pygmies populations). These differences in the life cycle have been usually associated with the amount of resources that an individual has or the presence of some individuals in their social network, but how the link of both of them change the life cycle of an individual remains unclear. Here I expect to show how the interplay of the amount of resources and sociality may be key to understand why the female human life cycle varies. For this, dynamic stochastic programming models will be used to analyse how different scenarios of the sociality surrounding resources varies the optimal female human life-history strategy. The results should demonstrate that variations in resources would play the main role in changes of the female human life cycle, while sociality would play a buffering effect of this oscilations. In conclusion, optimal female human life-history strategies vary because of the ways in which individuals have resources, being the interplay of the amount of resources and sociality a framework to understand it. This would help to understand why humans have evolved not only specific life-history traits but their variations as well.
\end{abstract}

\section{Introduction}

\begin{itemize}
    \item Introduction paragraph of the diversity of life-history strategies in humans
    \begin{itemize}
        \item Describe the human life cycle
        \item Describe how diverse it can be
        \item Describe the relationship of resources with the female human life cycle
        \item Describe the relationship of sociality with the female human life cycle
        \item Show that it is the interplay of sociality and resource the key to understand the diversity of life-history strategies
    \end{itemize}
    \item Overview of life-history theory and humans
    \begin{itemize}
        \item Introduce life-history theory
        \item Introduce human life-history theory
        \item Describe the main life-history trade-offs in humans
        \item Show that the field is obsessed with the quantity-quality trade-off
    \end{itemize}
    \item Overview of the competition-cooperation framework
    \begin{itemize}
        \item Describe the competition work
        \item Describe the cooperation work
        \item Show that it can be both
        \item Show that because of messy we want to focus not on this dichotomy but in how different sociality-resource scenarios can have different optimal life-history strategies
    \end{itemize}
    \item Sociality and resource transfers
    \begin{itemize}
        \item Brief overview of the work of kin presence
        \item State that it is not clear how they actually link
        \item Resource transfers can be a good way to explain how individuals get resources
        \item Brief overview of resource transfers in other species
        \item Brief overview of resource transfers in humans
        \item Resource transfers can be a good starting point to address sociality and link it to resources
    \end{itemize}
    \item Resources and the human life cycle
    \begin{itemize}
        \item Brief overview of how production and consumption shape the human life cycle
        \item Brief overview of resource accumulation
    \end{itemize}
    \item Conclusion paragraph
    \begin{itemize}
        \item Statement about how resources and sociality play a role in the life-history strategies of women
        \item Statement about how the interplay of both has not been properly addressed
        \item Present research question
        \item Brief overview of the paper
    \end{itemize}
\end{itemize}
\section{Methodology}

\begin{itemize}
    \item Models developed so far
    \begin{itemize}
        \item Brief overview of Kaplan's and Kramer's models
        \item Brief overview of Lee's model
        \item Statement that they approach the evolution of the average human life cycle, but not how it varies
        \item Statement where they don't considering resource accumulation
        \item Statement where resource transfers are a by-product
    \end{itemize}
    \item Dynamic stochastic programming models
    \begin{itemize}
        \item Brief description of dynamic stochastic programming models
        \item Examples of dynamic stochastic programming models in humans
        \item Statement of why this is the best model to answer our research question
    \end{itemize}
    \item Model description
    \begin{itemize}
        \item Currently working on this
    \end{itemize}
\end{itemize}

\section{Results}

\section{Discussion}

\bibliographystyle{apalike}
\bibliography{optimal_ref}

\end{document}
