\documentclass{article}
\usepackage[utf8]{inputenc}
\usepackage[margin=2.6cm]{geometry}
\usepackage{float}
\usepackage{rotating}
\usepackage{caption}
\usepackage[round]{natbib}
\usepackage{setspace}
\onehalfspacing

\title{List of parameters for model}
\author{Pablo Jose Varas Enriquez}
\date{\today}

\begin{document}

\maketitle

\section{Global parameters}

The global parameters of the model are the ones used to define the dynamics in each stage of the human life cycle.

\begin{itemize}
    \item Die: Probability of dying in the stage.
    \item Age: Amount of time spent in the stage.
    \item Produce: Probability of producing resources.
    \item Receive: Probability of receiving resources from another individual.
    \item Consume: Probability of consuming resources.
    \item Store: Probability of storing resources.
    \item Give: Probability of giving resources to another individual.
    \item Reproduce: Probability of producing offspring.
    \item Transition: Probability of transitioning to the next stage. The transitions are specified as follow:
    \begin{itemize}
        \item Infant transition: Reaching the growth and development necessary to transition to the juvenile stage.
        \item Sexual maturity: Reaching the development necessary for menarche and transition to the adult stage.
        \item Reproduce (Adult stage): Producing your first offspring and transition to the breeder stage.
        \item Menopause: Reaching the moment where you cannot, biologically, produce more offspring and transition to the post-reproductive stage.
    \end{itemize}
    \item Resources available: Amount of resources the individual has to allocate in storing, consuming, giving, survival, and/or transitioning.
    \item Resources stored: Amount of resources the individual stores for later in time.
\end{itemize}

\end{document}