\documentclass{article}
\usepackage[utf8]{inputenc}
\usepackage[margin=2.6cm]{geometry}
\usepackage{caption}
\usepackage[round]{natbib}
\usepackage{setspace}
\onehalfspacing

\title{The interplay of resources and sharing dynamics explains the diversity of female human reproductive decisions.}
\author{Pablo J. Varas Enriquez, Dieter Lukas, Heidi Colleran, Monique Borgerhoff Mulder $^*$\\\\
$*$ No special order of authors}
\date{\today}

\begin{document}

\maketitle

\begin{abstract}
    The female human life cycle is characterized by a long lifespan, within which there is usually a short reproductive period in between long juvenile and post-reproductive stages. The boundaries of the reproductive period, and the frequency at which births occur during this period, are influenced by individual reproductive decisions. Previous work has linked reproductive decisions to the amount of resources available to an individual in its environment or the availability of a network of potential helpers, but how the interplay of both of them change reproductive decisions of an individual remains unclear. Here we develop a theoretical framework to describe how the interplay of resource acquisition and their sharing across the social network may be key to understand female reproductive decisions. For this, an agent-based model was used to structure and describe how different resource and sharing dynamics, at different developmental stages (i.e. infant, juvenile, adult, breeder, and post-reproductive), can lead to different optimal reproductive decisions. We expect that differences in the ways in which individuals obtain different key resources and how these can be shared across the social network can explain the variability of the female life cycle in humans.
\end{abstract}

\end{document}