\documentclass{article}
\usepackage[utf8]{inputenc}
\usepackage[margin=2.6cm]{geometry}
\usepackage{caption}
\usepackage[round]{natbib}
\usepackage{setspace}
\onehalfspacing

\title{The interplay of resources and sociality explains the diversity of female human reproductive decisions.}
\author{Pablo J. Varas Enríquez, Dieter Lukas, Heidi Colleran, Monique Borgerhoff Mulder $^*$\\\\
$*$ No special order of authors}
\date{\today}

\begin{document}

\maketitle

\begin{abstract}
    The female human life cycle has evolved to have a long lifespan with a short reproductive period in between long juvenile and post-reproductive stages. Within these boundaries, individuals can show different  life cycles that can go from long lifespans and low reproductive output (e.g. post-demographic transition Japan) to short lifespans and high number of offspring (e.g. pygmies populations). These differences, based on differences in female reproductive decisions, are usually associated with the amount of resources that an individual has or the presence of some individuals in their social network, but how the link of both of them change the life cycle of an individual remains unclear. Here I expect to show a model that describes how the interplay of the amount of resources and sociality may be key to understand how female human reproductive decisions, and the associated life cycle, varies. For this, an agent-based model was used to analyse how different scenarios of the sociality surrounding resources varies the female reproductive decisions. The results demonstrate that variations in resources would play the main role in changes of the female human life cycle, while sociality would play a buffering effect of this oscillations. In conclusion, female reproductive decisions vary because of the ways in which individuals obtain resources, being the interplay of the amount of resources and sociality a framework to understand it. This would help to understand why female humans have evolved a specific life cycle, and in which conditions it varies.
\end{abstract}

\end{document}