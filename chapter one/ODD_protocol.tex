\documentclass{article}
\usepackage[utf8]{inputenc}
\usepackage[margin=2.6cm]{geometry}
\usepackage{float}
\usepackage{rotating}
\usepackage{caption}
\usepackage[round]{natbib}
\usepackage{setspace}
\onehalfspacing

\title{The interplay of resource acquisition and sharing dynamics explains the diversity of female human reproductive allocative decisions.
\\
ODD protocol}
\author{Pablo J. Varas Enriquez}
\date{\today}

\begin{document}

\maketitle

\tableofcontents

\section{Overview}

\subsection{Purpose}

The female human life cycle is characterised by a long lifespan, within which there is usually a short reproductive period in between long juvenile and post-reproductive stages. The boundaries of the reproductive period, and the frequency at which births occur during this period, are influenced by individual reproductive allocative decisions. Previous work has linked reproductive allocative decisions to the amount of resources available to an individual in its environment or the availability of a network of potential helpers, but how the interplay of both of them change reproductive allocative decisions of an individual remains unclear. Here I develop a theoretical framework to describe how the interplay of resource acquisition and their sharing across the social network may be key to understand female reproductive decisions. More specifically, it allows to analyse fitness differentials by simulating different scenarios in which the female human life cycle might vary. [Comment DL: why a agent-based simulation and not a sequence of equations: e.g. infant starts with 1 resource - probability to survive 0.85 * resources 1 - probability to receive 0.9 * 1 - 1.9 resources -> probability to consume 1 * 0.75 - 1.15 resources -> probability to store 0.95 * remaining - 1.0925 -> probability to loose 0.05 * remaining -> 1.038 resources on average at the end; if transition occurs at 1.2 resources means it takes 5.2 years to transition to juvenile; survive to that age 0.85*0.88*0.91*0.95*0.99 = 0.64]

\subsection{State variables and scales}

\subsubsection{State variables}

Every individual in the simulation is characterised by the following state variables:

\begin{itemize}
    \item Age: Sum of the amount of time spent in all stages.
    \item Stage: Life cycle stage in which the individual is at the moment.
    \item Resources produced: Amount of resources the individual produce, given the stage-specific production probability.
    \item Resources received: Amount of resources the individual receive, given the stage-specific receiving probability.
    \item Resources consumed: Amount of resources the individual consume for maintenance, given the stage-specific consumption probability.
    \item Resources gave: Amount of resources the individual gives, given the stage-specific giving probability.
    \item Resources stored: Amount of resources the individual stores for later in time, given the stage-specific storing probability.
    \item Resources lost: Amount of resources the individual loses, given the stage-specific losing probability.
    \item Resources available: Amount of resources the individual has to allocate in survival, reproduction, and/or transitioning, after the dynamics of producing, receiving, consuming, giving, storing, and losing resources.
    \item Reproduction: Sum of offspring produced, given the stage-specific fertility probability.
    \item Reproductive effort: Amount of resources the individual uses for reproduction, given the stage-specific probability.
\end{itemize}

\subsubsection{Auxiliary variables}

The individual dynamics are constrained by the following auxiliary variables:

\begin{itemize}
    \item Die: Probability of dying in the stage.
    \item Produce: Probability of producing resources.
    \item Receive: Probability of receiving resources from another individual.
    \item Times receiving: Number of times receiving resources from other individual(s).
    \item Consume: Stage-specific amount of resources necessary for somatic maintenance.
    \item Store: Probability of storing resources.
    \item Give: Probability of giving resources to another individual.
    \item Times giving: Number of times giving resources from other individual(s).
    \item Lose: Probability of losing resources.
    \item Reproduce: Probability of producing offspring.
    \item Transition: Probability of transitioning to the next stage. The transitions are specified as follow:
    \begin{itemize}
        \item Infant transition: Reaching the growth and development necessary to transition to the juvenile stage.
        \item Sexual maturity: Reaching the development necessary for menarche and transition to the adult stage.
        \item Reproduce (Adult stage): Producing your first offspring and transition to the breeder stage.
        \item Menopause: Reaching the moment where you cannot, biologically, produce more offspring and transition to the post-reproductive stage.
    \end{itemize}
\end{itemize}

\subsubsection{Higher entity description}

Individuals represent females in a single-sex, haploid population. Individuals that are born until they reach \emph{age at transition trait} are considered infants. Juveniles are individuals that transition from infants, until they reach age of sexual maturity. Adults are individuals that have reached sexual maturity until they have their first offspring or reach menopause. Adults transition to a breeder stage once they have their first offspring, and remain in this stage until they reach menopause. From menopause onward individuals are considered post-reproductive.

\subsubsection{Scale}

Each iteration in the model represents one year. The probabilities  of producing, receiving, consuming, giving, and losing are binomial, with the values of happening ($1$) or not ($0$). The amount of resources an individual produces, receives, consumes, gives, and lose are stage-specific. Reproductive effort is evaluated after the resource dynamics, followed by the evaluation for stage transition. Finally, the amount or resources available are passed from one year to the next, being evaluated for survival at the beginning of the iteration.

\subsection{Process overview and scheduling}

Infant individuals go through survival, receiving, consuming, storing, and losing sub models each year until they transition to the next stage. Once in the juvenile stage, individuals go through survive, produce, receive, consume, give, store, and lose sub models for each year until they reach sexual maturity, transitioning to the adult stage. In the adult stage, individuals also go through surviving, producing, receiving, consuming, giving, storing and losing sub models until they have their first offspring or reach menopause. If an adult transition to a breeder, it goes through surviving, producing, receiving, consuming, giving, storing and losing sub models and also through a reproduction sub model until it reaches menopause. Once in a post-reproductive stage, the individual goes through the surviving, producing, receiving, consuming, giving, storing and losing sub models of the stage. In each year, the individual increase its age and updates they amount of resources it has. During each transition, the individual updates its stage variable and also transitions with the resources it has available form the earlier stage.

\section{Design concepts}

\subsection{Emergence}

 The life cycle of an individual emerge from its behaviour. The life cycle is represented by the timing of its reproductive allocative decision and its reproductive output, constrained by the stage-specific submodels related to resource acquisition and sharing. Population dynamics also emerge from the behaviour of the individuals. The population dynamics are represented as the age distributions of the life cycle transitions, reproductive output, and surviving individuals. 

\subsection{Adaptation}

Not apply.

\subsection{Fitness}

Fitness is defined as population growth ($\lambda$).

\subsection{Prediction}

Not apply.

\subsection{Sensing}

Individuals are assumed to know their age, stage, and their current resources available  in order to apply the probabilities of producing, receiving, consuming, giving, losing resources, reproducing, surviving, and transitioning.

\subsection{Interaction}

Not apply.

\subsection{Stochasticity}

The life-history and resource dynamics models are based on probability distributions. Therefore, reproductive allostatic decisions are stochastic.

\subsection{Collectives}

Not apply.

\subsection{Observation}

For model testing, the life cycle of the individuals was observed process by process. For model analysis, timing (e.g. longevity, age of transition), reproductive (e.g. number of offspring), and resource-related (e.g. production, consumption, giving, receiving, losing) variables were recorded.

\section{Details}

\subsection{Initialisation}

At initialisation, each individual in the population starts in the infant stage with the baseline resources to survive, resembling the amount of resources acquired during pregnancy. Hence, an individual would be guaranteed to survive the first survival submodel and go through the resource dynamics submodels, only ageing after surviving the survival submodel in the next iteration.

\subsection{Input}

The resource dynamics that individuals are subjected to during their life cycle are drawn from binomial distributions that are specific for the life cycle stage in which the individual is.

\subsection{Submodels}

Each individual goes through all the stage-specific submodels each iteration.

\subsubsection{Infant}

\begin{itemize}
    \item Die: The infant survives by sampling a value equal or higher than the stage-specific survival rate from a *specify* distribution. The infant has a higher chance to survive if it has a larger amount of resources available. The amount of resources decreases by the costs of survival.
    \item Receive: The infant has a probability of receiving a fixed stage-specific amount of resources from another individual by sampling from a binomial distribution.
    \item Times receiving: The individual receives resources the number of times by randomly sampling a value between zero and the maximum number of individuals in the population.
    \item Consume: The infant consumes the amount of resources necessary for somatic maintenance.
    \item Store: The infant has a probability of storing a fixed stage-specific amount of resources by sampling from a binomial distribution.
    \item Lose: The infant has a probability of losing a fixed stage-specific amount of resources by sampling from a binomial distribution.
    \item Transition: The infant transition by sampling a value equal or higher than the stage-specific probability of transition from a *specify* distribution. The infant has a higher chance to transition if it has a larger amount of resources available. The amount of resources decreases by the costs of transition, while the remaining amount transition to the juvenile stage.
\end{itemize}

\subsubsection{Juvenile}

\begin{itemize}
    \item Die: The juvenile survives by sampling a value equal or higher than the stage-specific survival rate from a *specify* distribution. The juvenile has a higher chance to survive if it has a larger amount of resources available. The amount of resources decreases by the costs of survival.
    \item Produce: The juvenile has a probability of producing a fixed stage-specific amount of resources by sampling from a binomial distribution.
    \item Receive: The juvenile has a probability of receiving a fixed stage-specific amount of resources from another individual by sampling from a binomial distribution.
    \item Times receiving: The individual receives resources the number of times by randomly sampling a value between zero and the maximum number of individuals in the population.
    \item Consume: The juvenile consumes the amount of resources necessary for somatic maintenance.
    \item Give: The juvenile has a probability of giving a fixed stage-specific amount of resources to another individual by sampling from a binomial distribution.
    \item Times giving: The individual gives resources a number of times by randomly sampling a value between zero and the maximum number of individuals in the population.
    \item Store: The juvenile has a probability of storing a fixed stage-specific amount of resources by sampling from a binomial distribution.
    \item Lose: The juvenile has a probability of losing a fixed stage-specific amount of resources by sampling from a binomial distribution.
    \item Age at sexual maturity: The juvenile transition by sampling a value equal or higher than the stage-specific probability of transition from a *specify* distribution. The infant has a higher chance to be sexually mature if it has a larger amount of resources available. The amount of resources decreases by the costs of transition, while the remaining amount transition to the adult stage.
\end{itemize}

\subsubsection{Adult}

\begin{itemize}
    \item Die: The adult survives by sampling a value equal or higher than the stage-specific survival rate from a *specify* distribution. The adult has a higher chance to survive if it has a larger amount of resources available. The amount of resources decreases by the costs of survival.
    \item Produce: The adult has a probability of producing a fixed stage-specific amount of resources by sampling from a binomial distribution.
    \item Receive: The adult has a probability of receiving a fixed stage-specific amount of resources from another individual by sampling from a binomial distribution.
    \item Times receiving: The adult receives resources the number of times by randomly sampling a value between zero and the maximum number of individuals in the population.
    \item Consume: The adult consumes the amount of resources necessary for somatic maintenance.
    \item Give: The adult has a probability of giving a fixed stage-specific amount of resources to another individual by sampling from a binomial distribution.
    \item Times giving: The individual gives resources a number of times by randomly sampling a value between zero and the maximum number of individuals in the population.
    \item Store: The adult has a probability of storing a fixed stage-specific amount of resources by sampling from a binomial distribution.
    \item Lose: The adult has a probability of losing a fixed stage-specific amount of resources by sampling from a binomial distribution.
    \item Transition: The adult transition either to a reproductive stage (i.e. breeder) or a post-reproductive stage depending on either the individual have a offspring or reaches menopause, respectively.
    \begin{itemize}
        \item Age at first reproduction: The adult transition by sampling a value equal or higher than the stage-specific fertility rate from a Poisson distribution. The adult has a higher chance to have its first offspring if it has a larger amount of resources available. The amount of resources decreases by the costs of reproduction, while the remaining amount transition to the breeding stage.
        \item Menopause: The adult transition by sampling a value equal or higher than the stage-specific probability of transition from a *specify* distribution. The adult has a lower chance to transition if it has a larger amount of resources available, where the chance increases every iteration. The amount of resources decreases by the costs of menopause, while the remaining amount transition to the post-reproductive stage.
    \end{itemize}
\end{itemize}

\subsubsection{Breeder}

\begin{itemize}
    \item Die: The breeder survives by sampling a value equal or higher than the stage-specific survival rate from a *specify* distribution. The breeder has a higher chance to survive if it has a larger amount of resources available. The amount of resources decreases by the costs of survival.
    \item Produce: The breeder has a probability of producing a fixed stage-specific amount of resources by sampling from a binomial distribution.
    \item Receive: The breeder has a probability of receiving a fixed stage-specific amount of resources from another individual by sampling from a binomial distribution.
    \item Times receiving: The breeder receives resources the number of times by randomly sampling a value between zero and the maximum number of individuals in the population.
    \item Consume: The breeder consumes the amount of resources necessary for somatic maintenance.
    \item Give: The breeder has a probability of giving a fixed stage-specific amount of resources to another individual by sampling from a binomial distribution.
    \item Times giving: The individual gives resources a number of times by randomly sampling a value between zero and the maximum number of individuals in the population.
    \item Store: The breeder has a probability of storing a fixed stage-specific amount of resources by sampling from a binomial distribution.
    \item Lose: The breeder has a probability of losing a fixed stage-specific amount of resources by sampling from a binomial distribution.
   \item Reproduce: The breeder produce an offspring by sampling a value equal or higher than the stage-specific fertility rate from a Poisson distribution. The adult has a higher chance to have its first offspring if it has a larger amount of resources available. The amount of resources decreases by the costs of reproduction
    \item Menopause: The breeder transition by sampling a value equal or higher than the stage-specific probability of transition from a *specify* distribution. The breeder has a lower chance to transition if it has a larger amount of resources available, where the chance increases every iteration. The amount of resources decreases by the costs of menopause, while the remaining amount transition to the post-reproductive stage.
\end{itemize}

\subsubsection{Post-reproductive}

\begin{itemize}
    \item Die: The post-reproductive survives by sampling a value equal or higher than the stage-specific survival rate from a *specify* distribution. The individual has a higher chance to survive if it has a larger amount of resources available. The amount of resources decreases by the costs of survival.
    \item Produce: The post-reproductive has a probability of producing a fixed stage-specific amount of resources by sampling from a binomial distribution.
    \item Receive: The post-reproductive has a probability of receiving a fixed stage-specific amount of resources from another individual by sampling from a binomial distribution.
    \item Times receiving: The post-reproductive receives resources the number of times by randomly sampling a value between zero and the maximum number of individuals in the population.
    \item Consume: The post-reproductive consumes the amount of resources necessary for somatic maintenance.
    \item Give: The post-reproductive has a probability of giving a fixed stage-specific amount of resources to another individual by sampling from a binomial distribution.
    \item Times giving: The individual gives resources a number of times by randomly sampling a value between zero and the maximum number of individuals in the population.
    \item Store: The post-reproductive has a probability of storing a fixed stage-specific amount of resources by sampling from a binomial distribution.
    \item Lose: The post-reproductive has a probability of losing a fixed stage-specific amount of resources by sampling from a binomial distribution.
\end{itemize}

\end{document}
